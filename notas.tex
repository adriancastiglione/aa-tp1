%% LyX 2.3.0 created this file.  For more info, see http://www.lyx.org/.
%% Do not edit unless you really know what you are doing.
\documentclass[english]{article}
\usepackage[T1]{fontenc}
\usepackage[latin9]{inputenc}
\usepackage{geometry}
\geometry{verbose,tmargin=2cm,bmargin=2cm,lmargin=2cm,rmargin=2cm}
\usepackage{verbatim}

\makeatletter
%%%%%%%%%%%%%%%%%%%%%%%%%%%%%% Textclass specific LaTeX commands.
\providecommand*{\code}[1]{\texttt{#1}}

\makeatother

\usepackage{babel}
\begin{document}

\title{Notas para TP1}
\maketitle

\section{Separaci�n de datos para validaci�n.}

En principio hay que reservar instancias para el \emph{held-out. }De
los y tenemos 229 que son True (casi la mitad de 500). \\
(Si la proporci�n de True/False no fuera la mitad, y reservamos para
\emph{held-out} algunos con la mitad de True y la mitad de False;
no estar�amos sesgando al algoritmo?)\\

Opciones: 
\begin{itemize}
\item Separar al azar un porcentaje de las instancias
\item Ultimas o primeras
\end{itemize}
\begin{comment}
Por ahora no tenemos mucha informaci�n sobre qu� datos o atributos
son m�s importantes.. en principio esto decanta en separar de forma
random.
\end{comment}

\textbf{Importante: }Tenemos que tener en cuenta \emph{la competencia}.
En primer lugar tenemos 500 instancias disponibles, a dividir entre
entrenamiento y \emph{held-out; }mientras que los de competencia,
�nicamente para evaluaci�n, son 5000. \\
Si en un caso extremo tomamos los 500 datos para entrenamiento, igualmente
tendr�amos diez veces m�s datos sobre los cuales tendr�a que validarse
(\emph{competencia }como \emph{production}). Entonces, si tomamos
un porcentaje no-alto para entrenamiento, (por ejemplo reservando
$50\%$ para \emph{held-out}), se estar�a reflejando masomenos ver�dicamente
la situaci�n. \\
Tambi�n, el $50\%$ de \emph{held-out} puede reducirse gradualmente
si las pruebas de validaci�n dan horribles. De esa manera, guardamos
flexibilidad en la elecci�n de hiperpar�metros para ajustarlos despu�s.

\textbf{En resumen:} vamos a tomar, en principio, un $50\%$ de las
$500$ instancias para entrenamiento, y la otra mitad reservadas como
\emph{held-out.}

\section{Primeros modelos}

Primero creamos el �rbol de decisi�n vac�o con \code{max\-depth = 3}\\

\end{document}
